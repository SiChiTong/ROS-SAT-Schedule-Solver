\documentclass{article}
\title{Optimal Dub-E Scheduling}
\author{Skyler Peterson, Alex Sanchez-Stern}

\begin{document}
\maketitle
\section{Introduction}
For our final project,
we decided to look to robotics
for problems that SMT might be able to tackle.
Since we are fortunate enough to be at a school
where there is always interesting work going on across fields,
we didn't have to look far.
We contacted Michael Jae-Yoon Chung and Andrzej Pronobis,
who are working on the Semantics Aware Robotic Assistant,
more commonly known as DUB-E.
DUB-E is able to traverse the CSE building,
and accomplish tasks for it's users,
such as checking whether a particular professor is in their office.
DUB-E is controlled via a web interface,
from which users can request that certain tasks be accomplished,
by a particular deadline.

Unfortunately, when deadlines are short
and there are many tasks to accomplish,
it is non-trivial to decide what task
should be accomplished when.
Additionally, the expressive power of the interface to DUB-E
is currently limited;
Users can specify a tasks deadline,
but cannot specify more precise timing information,
such as a time in the future
before which the task should not be done,
or multiple time periods in which a task can be accomplished.

Handling these new scheduling concerns
requires a more sophisticated scheduling algorithm
than the one that was previously implemented in DUB-E.
We implemented this new scheduling algorithm
by encoding the scheduling constraints into SMT,
and then using Z3 as a backend
to solve the constraints.
The result is a schedule which instructs DUB-E
when to tackle each task and,
in some cases,
how long to wait in between tasks.
\section{Overview}
DUB-E's overall behavior is quite simple.
It is notified of new tasks via the web interface,
and it's scheduler decides what tasks to do when.
Each time it receives a new task,
it can change how it schedules it's current tasks.
Tasks which are completed are removed from it's schedule.

But not all tasks get completed.
If DUB-E is overscheduled,
it may not have enough time to complete a requested task in time.
Additionally, the travel time of DUB-E
is not entirely predictable.
While it's travel time between two locations
can mostly be bounded within a range,
it is always possible that it will completely fail,
and take much longer to reset itself
and become fully operational again.

The previous scheduling algorithm
for DUB-E was quite simple.
DUB-E simply maintained a FIFO queue
upon which it's tasks resided.
New requests made via the web interface
were added to the end of the queue,
and when DUB-E was not currently working on a task,
it pulled the next task from the front of the queue.
When DUB-E pulled a task from the queue
whose deadline had already passed,
it notified the user that it was unable to complete the task,
and dropped it from the queue.
While simple and fair,
this algorithm fails to provide
optimal behavior in a variety of simple scenarios.

Consider the case where user one
requests that DUB-E go the kitchen
and check for food
within the next thirty five minutes.
Let's assume that it takes DUB-E fifteen minutes
to complete this task,
ten minutes to go to the kitchen,
five to check for food.
The, user two requests that DUB-E
check whether Emina Torlak is in her office,
within the next twenty minutes.
It takes again, fifteen minutes total
for DUB-E to complete the task,
ten to arrive at Emina's office,
and five to confirm whether or not she is there.
If DUB-E addresses the tasks in a first come, first serve order,
as the old scheduler would do,
by the time it had finished checking for food,
it would have missed the deadline for finding Emina.
If instead it had reordered the tasks,
checking for Emina first,
it would be able to accomplish both tasks on time.

The DUB-E scheduling problem is not simply
a classic instance of scheduling
where each task takes a fixed amount of time.
Consider the case where DUB-E
is requested to check for both Emina and Zach Tatlock
in their respective offices.
It might be the case that both requests
should be completed in 18 minutes.
If DUB-E is in another part of the building,
it might take five minutes to get to both offices,
and take five minutes to check them.
Taken seperately, each request takes ten minutes,
and it is not possible to do both,
since it would take twenty minutes to do both individually.
But Zach's and Emina's offices are only a minute apart,
both being on the fifth floor,
it actually is possible to complete both tasks,
since one can go to one office, complete the task,
and travel to the other one,
in only eleven minutes.
Completing the second task takes a further five minutes,
for a total time of sixteen minutes,
well within the deadline.

It is clear from this example that
a proper scheduling algorithm must consider spatial factors,
as well as temporal ones.
Additionally, it may be the case that not all tasks can be scheduled,
and tasks have non-uniform priorities.
The DUB-E team asked for multiple task priorities,
that were weighted,
so that sufficiently many lower priority tasks
could be chosen over a higher priority task,
but also had an admin weight
that would never be dropped in favor of lower tasks.

\section{Scheduler Encoding}

To tackle these unique constraints,
we had to develop an encoding which could scale fairly well,
while still supporting all the features the DUB-E team wanted.
Since tasks can occur at a multitude of times,
and include times waited between tasks,
if a task is completed and the next one
belongs to an interval that has not started,
a naive approach would be to ask the solver
to simply give us the times at which each task is done,
given the problem constraints.
But solving for this many continious variables
would not scale to an even reasonable number of tasks.
Instead, we discretized time into an integer counter,
simplifying the issue even if we lose a little granularity,
and we divided the problem space into assignment of tasks
to ``time steps.''

For finding the optimal schedule of N tasks,
we attempt to find the best assignment
of tasks to N timesteps,
as well as solving for a single integer for each timestep
indicating how long should be waited before starting that timestep.
We constrain the solution to this in several ways
which allow results to be a valid schedule.

Since there exists a boolean variable for
each task being accomplished at each step,
we add one-hot constraints
so that at most one task is scheduled for each step,
and each task is scheduled for at most one step.
It might seem like we should require that
exactly one task be accomplished at each step,
and each task be accomplished at exactly one step.
However, it may be the case that
the tasks are over-constraining,
and so not all of them can be accomplished.
To account for this case,
we must allow for tasks to be not scheduled for any step,
and for some steps to be empty.
We create a boolean variable for each timestep t labeled None@t,
which indicates whether or not \textbf{no} tasks are scheduled at that timestep,
and require that exactly one of the set of $\{None\} \cup tasksScheduled$
be true for each timestep.

To make other constraints,
as well as the resulting schedule,
simpler,
we also require that all the None's
are at the end of the schedule.
That is, if a timestep has None,
then all timesteps after it have None,
forcing all actual tasks to the front of the timesteps.

Now that we've encoded
the notion of assignment of tasks to timesteps
into constraints,
we must encode the timing constraints.
But how long does traveling between tasks task?
Due to the unpredictability of the robot and it's environment,
we don't have a fixed bound on how long traveling between locations will take.
However, we do have a notion of the longest they should take
if nothing goes terribly wrong.
To simplify the scheduling,
we decided to ask the scheduler to only produce
schedules where DUB-E will always have enough travel time
if nothing goes terribly wrong.
If something \textit{does} go terribly wrong,
and DUB-E falls behind schedule,
we reschedule.
Here, we slightly comprimise on optimality,
but we hope that the failure rate
will be low enough that
it won't be significantly worse in practice.

Once we have decided on an effective travel time,
we create two Int variable for each timestep.
The first is the previously mentioned waitBefore,
the amount of time waited after finishing the last step
before starting the next one.
This variable is constrained to be non-negative,
but is otherwise free to be assigned whatever value
makes the schedule work out best.

The second Int variable
is the time at which that step is started.
This variable is constrained
to be exactly equal to the time
of the previous time variable,
plus the waitBefore time,
plus the time taken to complete the previous task,
plus the time taken to travel
from the previous task to the next task.
We encode the time taken to complete the previous task
using the ``ite'' construct for each possible previous task,
giving a zero value if that task was not completed last timestep,
and the value of the task duration if it was.
We similarly encode the travel time,
using every possible pair of tasks we could have traveled in between.
Next, we require that if a task is done at a particular timestep,
we require that that timestep be fully within one of the tasks intervals.
That means that the task must start after the beginning of the interval,
and be finished before the end.

Finally, once we have properly constrained the solver
so that every solution will be a valid schedule,
that the robot can execute,
we attempt to find the \textit{optimal} schedule.
We decided to encode task weights as integers,
where higher is better,
but zero is an admin task.

First, we create a variable
for each task being accomplished
at some timestep,
and set up an implication constraint
so that it can only be true
if the the task was completed at some timestep.
Then, we use a MaxSAT algorithm
to determine the schedule which can fit in the most admin tasks.
Then, we run it again to find the schedule with the greatest
sum weights of non-admin tasks,
given that it must accomplish the admin tasks found.
We reimplemented
the WMax algorithm described here%
\footnote{http://research.microsoft.com/en-US/people/nbjorner/scss2014.pdf},
because we were not able to find the tool itself.
This approach is slightly flawed,
since there may be multiple optimal sets of admin tasks,
and one may allow more non-admin tasks,
but we felt that the slight benefits
of trying multiple configurations of optimal admin tasks
did not outweigh the running time cost.

Once we've figured out the optimal configuration,
we parse the list booleans
indicating whether each task was accomplished at each timestep,
plus the waitBefore times of each step,
into a path, and give it to the ROS interface.

\section{Results}

\section{Project Division}
We roughly divided the work into the encoding,
and the interface with DUB-E.
Skyler handled the interface with the robot,
setting up ROS nodes and working with Michael and Andrzej
to get the encoder integrated
into the DUB-E codebase.
Alex handled the actual encoding of the task constraints,
setting up the Z3 bindings to interface with the code,
and writing modules to take in task information,
encode it into a set of Z3 constraints,
decode the Z3 output into a schedule,
and format the schedule for DUB-E.

\section{Applied Topics}
This project involved mostly topics
that we discussed at the beginning of the quarter,
although it of course was supported by
the work we did throughout the quarter.
Specifically, the work on encoding different types of constraints
into conjunctive normal form was paramount to the project.
The encoding also made heavy use of MaxSAT,
which we touched on in class,
to satisfy the greatest number of tasks
in cases where not all tasks could be satisfied.
\end{document}
